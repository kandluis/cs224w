\documentclass[11pt]{article}
\usepackage{amsmath}
\usepackage{amssymb}
\usepackage{graphicx}
\usepackage{fancyhdr}
\usepackage{enumerate}
\usepackage{titlesec}
\usepackage[colorlinks=true,urlcolor=blue]{hyperref}

\titlespacing{\subsubsection}{0pt}{0pt}{0pt}

% No page numbers
%\pagenumbering{gobble}

% INFORMATION SHEET (DO NOT EDIT THIS PART) ---------------------------------------------
\newcommand{\addinformationsheet}{
\clearpage
\thispagestyle{empty}
\begin{center}
\LARGE{\bf \textsf{Information sheet\\CS224W: Analysis of Networks}} \\*[4ex]
\end{center}
\vfill
\textbf{Assignment Submission } Fill in and include this information sheet with each of your assignments.  This page should be the last page of your submission.  Assignments are due at 11:59pm and are always due on a Thursday.  All students (SCPD and non-SCPD) must submit their homeworks via GradeScope (\url{http://www.gradescope.com}). Students can typeset or scan their homeworks. Make sure that you answer each (sub-)question on a separate page. That is, one answer per page regardless of the answer length. Students also need to upload their code at \url{http://snap.stanford.edu/submit}. Put all the code for a single question into a single file and upload it. Please do not put any code in your GradeScope submissions. 
\\
\\
\textbf{Late Homework Policy } Each student will have a total of {\em two} free late periods. {\em Homeworks are due on Thursdays at 11:59pm PDT and one late period expires on the following Monday at 11:59pm PDT}.  Only one late period may be used for an assignment.  Any homework received after 11:59pm PDT on the Monday following the homework due date will receive no credit.  Once these late periods are exhausted, any assignments turned in late will receive no credit.
\\
\\
\textbf{Honor Code } We strongly encourage students to form study groups. Students may discuss and work on homework problems in groups. However, each student must write down their solutions independently i.e., each student must understand the solution well enough in order to reconstruct it by him/herself.  Students should clearly mention the names of all the other students who were part of their discussion group. Using code or solutions obtained from the web (github/google/previous year solutions etc.) is considered an honor code violation. We check all the submissions for plagiarism. We take the honor code very seriously and expect students to do the same. 
\vfill
\vfill
}
% ------------------------------------------------------------------------------

% MARGINS (DO NOT EDIT) ---------------------------------------------
\oddsidemargin  0.25in \evensidemargin 0.25in \topmargin -0.5in
\headheight 0in \headsep 0.1in
\textwidth  6.5in \textheight 9in
\parskip 1.25ex  \parindent 0ex \footskip 20pt
% ---------------------------------------------------------------------------------

% HEADER (DO NOT EDIT) -----------------------------------------------
\newcommand{\problemnumber}{0}
\newcommand{\myname}{name}
\newfont{\myfont}{cmssbx10 scaled 1000}
\pagestyle{fancy}
\fancyhead{}
\fancyhead[L]{\myfont Question \problemnumber, Problem Set 2, CS224W}
%\fancyhead[R]{\bssnine \myname}
\newcommand{\newquestion}[1]{
\clearpage % page break and flush floats
\renewcommand{\problemnumber}{#1} % set problem number for header
\phantom{}  % Put something on the page so it shows
}
% ---------------------------------------------------------------------------------


% BEGIN HOMEWORK HERE
\begin{document}
\graphicspath{ {../../code/output/} }

% Question 1.1
\newquestion{1.1}

\subsubsection*{(a)}
We draw 100 random (simple) network samples using the stub-matching algorithm described in the assignment using the degree sequence of the power grid network. 

The mean of the average clustering coefficient for 100 random networks with the same degree distribution as the power grid network is 0.000427366228043.

\subsubsection*{(b)}
We now consider node $i$ with degree $k_i$, and node $j$ with degree $k_j$. Under a random matching on the half-edges, the probability that node $i$ and node $j$ are connected is given by:
$$
p_{i,j} = \frac{k_ik_j}{\sum_{l} k_l - 1} =  \frac{k_ik_j}{2|E| - 1}
$$
The above can be derived as follows. Allowing self-loops and multi-edges, for each of the $k_i$ stubs of node $i$, the probability that a specific one matches to one of the $k_j$ stubs of node $j$ is $\frac{k_j}{2|E| - 1}$ since there are $k_j$ matches and $2|E| - 1$ candidates, and the selection is random. Since we do allow multi-edges and self-loops, each event is independent, therefore the probability of any such event occurring is simple the sum of all $k_i$ events, which by symmetry, occur with the same probability, giving us the above formula.

The above immediately implies that the higher the degrees of $i$ and $j$, the more likely they are to be connected under the configuration model.

\subsubsection*{(c)}
We show that artificially rejecting self-loops when building the graph is not a problem for large graphs. We consider a graph with a very large number of edges m, and calculate the number of expected self-loops, $X$, as follows:

\begin{align*}
E[X] &= \sum_{i=1}^n E[X_i] \tag{where $X_i$ is the number of self-loops for node $i$} \\
&= \sum_{i=1}^n {k_i \choose 2} \frac{1}{2m - 1} \tag{we chose two of $k_i$ stubs and calculate the probability that one connects to the other} \\
&= \sum_{i=1}^n \frac{k_i(k_i - 1)}{4m} \\
&= \frac{1}{2}\frac{1}{2m} \left(\sum_{i=1}^n k_i^2 - \sum_{i=1}^n k_i\right) \tag{rewriting} \\
&= \frac{1}{2}\frac{n}{2m} \left(\frac{1}{n}\sum_{i=1}^n k_i^2 - \frac{1}{n}\sum_{i=1}^n k_i\right) \tag{multiply by $\frac{n}{n}$} \\
&= \frac{1}{2}\frac{n}{\sum_{i=1}^n k_i } \left(\frac{1}{n}\sum_{i=1}^n k_i^2 - \frac{1}{n}\sum_{i=1}^n k_i\right) \tag{using the fact that $\sum_{i=1}^n k_i = 2m$} \\
&= \frac{\langle k^2 \rangle - \langle k \rangle}{2\langle k \rangle} \tag{using $\langle k^m \rangle = \frac{1}{n}\sum_{i=1}^n k_i^m$}
\end{align*}
The above result implies that the expected number of self-loops is vanishingly small $O(\frac{1}{n})$ fraction of all edges in the large $n$ limit as long as $\langle k^2 \rangle$ is finite. This means we expect to see no self-loops in the limit.

% Question 1.2
\newquestion{1.2}
We plot the average clustering coefficient as a function of the iteration number for
the rewiring algorithm.

\begin{figure}[h!]
\centering
\includegraphics[width=0.7\textwidth]{1_2_plot}
\caption{Plot of average clustering coefficient as we transform the power grid network into a random network with the same degree distribution.}
\label{fig:power_grid_to_configuration}
\end{figure}

We note from the above graph that the average clustering coefficient for the power grid network is ``high'' (by a factor of 200) with respect to what we would expect from just a random network. We can see this as we rewire the edges to approach a random network, since the average clustering coefficient decreases from 0.08010361108159704 to 0.00041768990958905632 (average of the last 5 values), approximately the same clustering coefficient as the random network we generated in in (a) above. 

% Question 2.1
\newquestion{2.1}

\subsubsection*{(a)}

\subsubsection*{(b)}

\subsubsection*{(c)}

% Question 2.2
\newquestion{2.2}

\subsubsection*{(a)}

\subsubsection*{(b)}

\subsubsection*{(c)}

\subsubsection*{(d)}

% Question 2.3
\newquestion{2.3}

% Question 2.4
\newquestion{2.4}

% Question 2.5
\newquestion{2.5}

% Question 2.6
\newquestion{2.6}

% Question 3.1
\newquestion{3.1}

% Question 3.2
\newquestion{3.2}

% Question 3.3
\newquestion{3.3}

% Question 3.4
\newquestion{3.4}

% Information sheet
% Fill out the information below (this should be the last page of your assignment)
\addinformationsheet
{\Large
\textbf{Your name:} \hrulefill  % Put your name here
\\
\textbf{Email:} \underline{\hspace*{7cm}}  % Put your e-mail here
\textbf{SUID:} \hrulefill  % Put your student ID here
\\*[2ex] 
}
Discussion Group: \hrulefill   % List your study group here
\\
\vfill\vfill
I acknowledge and accept the Honor Code.\\*[3ex]
\bigskip
\textit{(Signed)} 
\hrulefill   % Replace this line with your initials
\vfill





\end{document}
