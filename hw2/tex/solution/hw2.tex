\documentclass[11pt]{article}
\usepackage{amsmath}
\usepackage{amssymb}
\usepackage{graphicx}
\usepackage{fancyhdr}
\usepackage{enumerate}
\usepackage{titlesec}
\usepackage[english]{babel}
\usepackage[utf8x]{inputenc}
\usepackage{tikz}
\usetikzlibrary{arrows,automata}
\usepackage[colorlinks=true,urlcolor=blue]{hyperref}

\titlespacing{\subsubsection}{0pt}{0pt}{0pt}

% No page numbers
%\pagenumbering{gobble}

% INFORMATION SHEET (DO NOT EDIT THIS PART) ---------------------------------------------
\newcommand{\addinformationsheet}{
\clearpage
\thispagestyle{empty}
\begin{center}
\LARGE{\bf \textsf{Information sheet\\CS224W: Analysis of Networks}} \\*[4ex]
\end{center}
\vfill
\textbf{Assignment Submission } Fill in and include this information sheet with each of your assignments.  This page should be the last page of your submission.  Assignments are due at 11:59pm and are always due on a Thursday.  All students (SCPD and non-SCPD) must submit their homeworks via GradeScope (\url{http://www.gradescope.com}). Students can typeset or scan their homeworks. Make sure that you answer each (sub-)question on a separate page. That is, one answer per page regardless of the answer length. Students also need to upload their code at \url{http://snap.stanford.edu/submit}. Put all the code for a single question into a single file and upload it. Please do not put any code in your GradeScope submissions. 
\\
\\
\textbf{Late Homework Policy } Each student will have a total of {\em two} free late periods. {\em Homeworks are due on Thursdays at 11:59pm PDT and one late period expires on the following Monday at 11:59pm PDT}.  Only one late period may be used for an assignment.  Any homework received after 11:59pm PDT on the Monday following the homework due date will receive no credit.  Once these late periods are exhausted, any assignments turned in late will receive no credit.
\\
\\
\textbf{Honor Code } We strongly encourage students to form study groups. Students may discuss and work on homework problems in groups. However, each student must write down their solutions independently i.e., each student must understand the solution well enough in order to reconstruct it by him/herself.  Students should clearly mention the names of all the other students who were part of their discussion group. Using code or solutions obtained from the web (github/google/previous year solutions etc.) is considered an honor code violation. We check all the submissions for plagiarism. We take the honor code very seriously and expect students to do the same. 
\vfill
\vfill
}
% ------------------------------------------------------------------------------

% MARGINS (DO NOT EDIT) ---------------------------------------------
\oddsidemargin  0.25in \evensidemargin 0.25in \topmargin -0.5in
\headheight 0in \headsep 0.1in
\textwidth  6.5in \textheight 9in
\parskip 1.25ex  \parindent 0ex \footskip 20pt
% ---------------------------------------------------------------------------------

% HEADER (DO NOT EDIT) -----------------------------------------------
\newcommand{\problemnumber}{0}
\newcommand{\myname}{name}
\newfont{\myfont}{cmssbx10 scaled 1000}
\pagestyle{fancy}
\fancyhead{}
\fancyhead[L]{\myfont Question \problemnumber, Problem Set 2, CS224W}
%\fancyhead[R]{\bssnine \myname}
\newcommand{\newquestion}[1]{
\clearpage % page break and flush floats
\renewcommand{\problemnumber}{#1} % set problem number for header
\phantom{}  % Put something on the page so it shows
}
% ---------------------------------------------------------------------------------


% BEGIN HOMEWORK HERE
\begin{document}
\graphicspath{ {../../code/output/} }

% Question 1.1
\newquestion{1.1}

\subsubsection*{(a)}
We draw 100 random (simple) network samples using the stub-matching algorithm described in the assignment using the degree sequence of the power grid network. 

The mean of the average clustering coefficient for 100 random networks with the same degree distribution as the power grid network is 0.000427366228043.

\subsubsection*{(b)}
We now consider node $i$ with degree $k_i$, and node $j$ with degree $k_j$. Under a random matching on the half-edges, the probability that node $i$ and node $j$ are connected is given by:
$$
p_{i,j} = \frac{k_ik_j}{\sum_{l} k_l - 1} =  \frac{k_ik_j}{2|E| - 1}
$$
The above can be derived as follows. Allowing self-loops and multi-edges, for each of the $k_i$ stubs of node $i$, the probability that a specific one matches to one of the $k_j$ stubs of node $j$ is $\frac{k_j}{2|E| - 1}$ since there are $k_j$ matches and $2|E| - 1$ candidates, and the selection is random. Since we do allow multi-edges and self-loops, each event is independent, therefore the probability of any such event occurring is simple the sum of all $k_i$ events, which by symmetry, occur with the same probability, giving us the above formula.

The above immediately implies that the higher the degrees of $i$ and $j$, the more likely they are to be connected under the configuration model.

\subsubsection*{(c)}
We show that artificially rejecting self-loops when building the graph is not a problem for large graphs. We consider a graph with a very large number of edges m, and calculate the number of expected self-loops, $X$, as follows:

\begin{align*}
E[X] &= \sum_{i=1}^n E[X_i] \tag{where $X_i$ is the number of self-loops for node $i$} \\
&= \sum_{i=1}^n {k_i \choose 2} \frac{1}{2m - 1} \tag{we chose two of $k_i$ stubs and calculate the probability that one connects to the other} \\
&= \sum_{i=1}^n \frac{k_i(k_i - 1)}{4m} \\
&= \frac{1}{2}\frac{1}{2m} \left(\sum_{i=1}^n k_i^2 - \sum_{i=1}^n k_i\right) \tag{rewriting} \\
&= \frac{1}{2}\frac{n}{2m} \left(\frac{1}{n}\sum_{i=1}^n k_i^2 - \frac{1}{n}\sum_{i=1}^n k_i\right) \tag{multiply by $\frac{n}{n}$} \\
&= \frac{1}{2}\frac{n}{\sum_{i=1}^n k_i } \left(\frac{1}{n}\sum_{i=1}^n k_i^2 - \frac{1}{n}\sum_{i=1}^n k_i\right) \tag{using the fact that $\sum_{i=1}^n k_i = 2m$} \\
&= \frac{\langle k^2 \rangle - \langle k \rangle}{2\langle k \rangle} \tag{using $\langle k^m \rangle = \frac{1}{n}\sum_{i=1}^n k_i^m$}
\end{align*}
The above result implies that the expected number of self-loops is vanishingly small $O(\frac{1}{n})$ fraction of all edges in the large $n$ limit as long as $\langle k^2 \rangle$ is finite. This means we expect to see no self-loops in the limit.

% Question 1.2
\newquestion{1.2}
We plot the average clustering coefficient as a function of the iteration number for
the rewiring algorithm.

\begin{figure}[h!]
\centering
\includegraphics[width=0.7\textwidth]{1_2_plot}
\caption{Plot of average clustering coefficient as we transform the power grid network into a random network with the same degree distribution.}
\label{fig:power_grid_to_configuration}
\end{figure}

We note from the above graph that the average clustering coefficient for the power grid network is ``high'' (by a factor of 200) with respect to what we would expect from just a random network. We can see this as we rewire the edges to approach a random network, since the average clustering coefficient decreases from 0.08010361108159704 to 0.00041768990958905632 (average of the last 5 values), approximately the same clustering coefficient as the random network we generated in in (a) above. 

% Question 2.1
\newquestion{2.1}

\subsubsection*{(a)}
Count of Triad t0: 58732

Count of Triad t1: 396548

Count of Triad t2: 451711

Count of Triad t3: 4003085

Fraction of Triad t0: 0.0120

Fraction of Triad t1: 0.0808

Fraction of Triad t2: 0.0920

Fraction of Triad t3: 0.8153

\subsubsection*{(b)}
Fraction of Positive Edges: 0.8324

Fraction of Negative Edges: 0.1676

Probability of Triad t0: 0.0047

Probability of Triad t1: 0.0701

Probability of Triad t2: 0.3484

Probability of Triad t3: 0.5768

\subsubsection*{(c)}
Comparing (a) and (b), we see more t0 and t3 and fewer t2 triads (approximately expected t1, though slightly more) in the Epinions network then we would expect based on chance given the proportion of positive and negative edges.

One possible explanation is that the Epinions network is quite social, so many of the t2 triads are quite unstable (I trust two people that do not trust each other), and often change into a fully trusting triad (t3), explaining the decrease in t2 and dramatic increase in t3 triads compared to random.

As for the slight increase in t0, it might be due to the radical nature of individuals. Any level of untrustworthiness poisons the relationships. It also implies that the corollary ``I will trust someone who doesn't trust someone I don't trust'' does not hold in this network. This makes some sense for ``trust''since an untrustworthy person not trusting someone else could imply the person is more untrustworthy than the first.

% Question 2.2
\newquestion{2.2}

\subsubsection*{(a)}
Let $T$ be a maximum set of disjoint-edge triangles in $G$, a complete graph. We give a simple lower bound for $|T|$.

$$
\left\lfloor \frac{n - 1}{2} \right\rfloor \leq |T|
$$
We proof the above by induction on $n$, the number of nodes in the graph. Consider $n=1$ and $n=2$ as our base cases and note that:
\begin{align*}
\left\lfloor \frac{1 - 1}{2} \right\rfloor = 0 \leq |\{\}| \\
\left\lfloor \frac{2 - 1}{2} \right\rfloor = 0 \leq |\{\}| \\
\end{align*}

Now assume the above holds for some $G_i$ with $i$ nodes. Let $T_i$ be the maximum set of disjoin-edge triangles in $G_i$. We prove that it holds for $k = i + 2$. Consider $G_k$ with two distinct nodes (and corresponding edges) removed, and note this graph is isomorphic to $G_i$. Let the two nodes be $u,v$. Consider a third node $w \neq u \neq v$. Note that $(u,v,w)$ form a disjoint-edge triangle with $|T_i|$ since no edge is contained in $G_i$. Therefore:

\begin{align*}
\left\lfloor \frac{k - 1}{2} \right\rfloor = \left\lfloor \frac{i + 1}{2} \right\rfloor = \left\lfloor \frac{i - 1}{2} + 1 \right\rfloor &= \left\lfloor \frac{i - 1}{2}\right\rfloor + 1 \\
&\leq |T_i| + 1 \tag{by induction} \\
&= |T_i \cup \{(u,v,w\}| \tag{disjoin sets}\\
&\leq |T_{i+1}| = |T_k| \tag{the maximum could only be larger than one example}
\end{align*}


Note that the above together with $n=0$ proves our claim for even $n$, and with $n=1$ proves our claim for odd $n$. 

\subsubsection*{(b)}
A triangle in $G$ is balanced if all its edges are positive or only one edge is positive. Therefore the probability that a triangle, $t$, is balanced is given by:
$$
P(t_B) = p^3 + 3p(1-p)^2
$$
and with $p = \frac{1}{2}$ we have the probability as $P(t_B) = \frac{1}{2}$.

\subsubsection*{(c)}
We give an upper bound on the probability that all of the triangles in $T$ are balanced and show that this probability approaches $0$ as $n \to \infty$. Let $T_B$ be the event that all the triangles in $T$ are balanced. Then we have:

\begin{align*}
P(T_B) &= \bigwedge_{t \in T} P(t_B) \tag{triangles are disjoint, therefore events independent} \\
&= \left[p^3 + 3p(1-p)^2\right]^{|T|} \tag{results from (b)} \\
&= \left[p^3 + 3p(1-p)^2\right]^{\left\lfloor \frac{n - 1}{2} \right\rfloor} \tag{results from (a), noting that probabilities are $\leq 1$} \\
&\leq \left(\frac{1}{2}\right)^{\left\lfloor \frac{n - 1}{2} \right\rfloor} \tag{assuming $p = \frac{1}{2}$} \\
&= O\left(\frac{1}{2^n}\right)
\end{align*}

From the above, we can immediately see that $n \to \infty$, $P(T_B) \to 0$. Note that this further holds in the general case, since for all $0 \leq p < 1$ we have that $p^3 + 3p(1-p)^2 < 1$ which is sufficient for $O(p^n) \to 0$. Note that for $p = 1$ our results do not hold. Intuitively, when $p = 1$, we always have a balanced network since all edges are positive.

\subsubsection*{(d)}
Part (c) further implies that $P(G_B) \to 0$ as $n \to \infty$ since $T$ forms a subgraph of $G$, and in order for $G_B$ to occur, every subgraph of $G$ must be balanced.

% Question 2.3
\newquestion{2.3}
In the process described, it is not true that the number of balanced triads cannot decrease. Consider the following counter example on a complete graph $G$ with $5$ nodes. Let us consider the subgraph $G'$ of $G$ shown below:

\begin{centering}
\begin{tikzpicture}[-,shorten >=3pt,auto,node distance=5.5cm,
        scale = 1,transform shape]
  \node[state] (1) {$1$};
  \node[state] (2) [right of=1] {$2$};
  \node[state] (4) [above right of=2] {$4$};
  \node[state] (3) [below right of=4] {$3$};
  \node[state] (5) [below right of=2] {$5$};

  \path (1) edge              node {$+$} (4)
        (1) edge              node {$+$} (5)
        (2) edge              node {$+$} (4)
        (2) edge              node {$+$} (5)
        (4) edge              node {$+$} (5)
        (3) edge              node {$+$} (4)
        (3) edge              node {$-$} (5);

\end{tikzpicture}
\end{centering}

First, note that we have a total of 3 triads with 2 balanced $\{(1,4,5), (2,4,5)\}$ and 1 unbalanced $(3,4,5)$.

Now suppose we randomly choose the unbalanced triad in our dynamic process. Furthermore, suppose we select edge $(4,5)$ as the one to flip in order to balance $(3,4,5)$. Then note that the new graph will be:

\begin{centering}
\begin{tikzpicture}[-,shorten >=3pt,auto,node distance=5.5cm,
        scale = 1,transform shape]
  \node[state] (1) {$1$};
  \node[state] (2) [right of=1] {$2$};
  \node[state] (4) [above right of=2] {$4$};
  \node[state] (3) [below right of=4] {$3$};
  \node[state] (5) [below right of=2] {$5$};

  \path (1) edge              node {$+$} (4)
        (1) edge              node {$+$} (5)
        (2) edge              node {$+$} (4)
        (2) edge              node {$+$} (5)
        (4) edge              node {$-$} (5)
        (3) edge              node {$+$} (4)
        (3) edge              node {$-$} (5);

\end{tikzpicture}
\end{centering}

However, now we only have 1 balanced triad $(3,4,5)$ and the other two have become unbalanced. Therefore, this is a counterexample demonstrating that the process outlined does not always increase the number of balanced triads.

% Question 2.4
\newquestion{2.4}
100\% of the trials ended up with balanced networks.

% Question 2.5
\newquestion{2.5}
Let us define the following function:
\[
  \text{sign}(u,v) =
  \begin{cases}
                                   1 & \text{if $(u,v)=+$} \\
                                   0 & \text{if $(u,v)=-$}
  \end{cases}
\]

It is not possible. Note that adding a node $D$ will form three new edges, and subsequently, three new triangles - $(A,B,D), (A,C,D)$ and $(B,C,D)$.

Suppose that all of these triangles are balanced. Particularly, note that this implies that $\text{sign}(B,D) = 1 - \text{sign}(C,D)$ since the only balanced triangle with a negative edge has another negative and positive edge (triangle $(B,C,D)$). Looking at triangles ($A,B,D$) and $(A,C,D)$, assuming they are balanced, we must have $\text{sign}(A,D) = \text{sign}(B,D)$ and $\text{sign}(A,D) = \text{sign}(C,D)$ since either the triangle is all positive or the new edges are both negative.

However, note that the above leads to the following:
$$
\text{sign}(B,D) = 1 - \text{sign}(C,D) = 1 - \text{sign}(A,D) = 1 - \text{sign}(B,D) \implies 0 = 1
$$
a contradiction. Therefore all triangles cannot be balanced.
% Question 2.6
\newquestion{2.6}
Take any complete \text{sign}ed network, on any number of nodes, that is unbalanced. We show that it is not possible for a new node X to join the network and form edges to all existing nodes in such a way that it does not become involved in any unbalanced triangles. 

First, we consider a problem similar to 2.5 where we have a complete graph on $\{A,B,C\}$ where all \text{sign}s are negative (the only triangle is unbalanced). Let us consider adding a node $D$ to this graph. Assuming that the three newly crated triangles involving $D$ are balanced, we must have $\text{sign}(D,A) = 1 - \text{sign}(D,B), \text{sign}(D,B) = 1 -\text{sign}(D,C)$ and $\text{sign}(D,C) = 1 -\text{sign}(D,A)$. Note that putting these equations together we have:
$$
\text{sign}(D,A) = 1 - \text{sign}(D,B) = -\text{sign}(D,C) = \text{sign}(D,A) - 1 \implies 0 = 1
$$
a contradiction. Therefore all triangles involving $D$ cannot be balanced.

With the above and 2.5, we have now shown that adding a new node $X$ to a currently unbalanced triangle will necessarily form at least one new unbalanced triangle in which $X$ participates. This readily generalizes to any complete signed network on any number of nodes.

Take $G = (V,E)$ to be a complete signed network on $n$ nodes that is unbalanced. Since $G$ is unbalanced, there must exists $(u,v,w) \in V$ such that the triangle formed by $u,v,w$ is unbalanced. Now consider $X$ joining $G$ and forming an edge to every node in $G$. Then it must form edges to each of the nodes $\{u,v,w\}$. By our previous arguments, this implies $X$ must participate in an unbalanced triangle.

% Question 3.1
\newquestion{3.1}
Candidate B wins by 76 votes in Graph 1.
Candidate B wins by 350 votes in Graph 2.
% Question 3.2
\newquestion{3.2}
We plot the winning margin as a function of ad-money spent:

\begin{figure}[h!]
\centering
\includegraphics[width=0.7\textwidth]{3_2_plot}
\caption{Plot of the margin of victory for A as a function of money spent on ads}
\label{fig:admoney}
\end{figure}

We can spend about \$5000 (based on our experiments) and win the election on Graph 1 in favor of A by 36 votes.
We can spend about \$7000 (based on our experiments) and win the election on Graph 2 in favor of A by 24 votes.

% Question 3.3
\newquestion{3.3}
We plot the winning margin as a function of money spent dining high rollers:

\begin{figure}[h!]
\centering
\includegraphics[width=0.7\textwidth]{3_3_plot}
\caption{Plot of the margin of victory for A as a function of money spent on high rollers}
\label{fig:high_rollers}
\end{figure}

We can spend \$6000 and win the election on Graph 2 in favor of A by 142 votes.
By this method, we are unable to swing the election in favor of A with just a \$9000 budget in Graph 1.

% Question 3.4
\newquestion{3.4}
We show the log-log plot of the degree distributions for Graph 1 and Graph 2.

\begin{figure}[h!]
\centering
\includegraphics[width=0.7\textwidth]{3_4_plot}
\caption{Degree distribution for Graph 1 and Graph 2}
\label{fig:distribution}
\end{figure}

From the above, it is obvious that the random graph is Graph 1 -- it has the distribution we have seen for the ER-model.

The above explains the results in 3.3. The ER-model (Graph 1) has high-rollers with very low-degree, so changing their minds affects only a small set of nodes in the network, therefore it is not an effective way to convince nodes to vote for A. However, we can see that the high-rollers for Graph 2 have a very high degree, therefore capture just a few of them can influence many nodes in the network.

% Information sheet
% Fill out the information below (this should be the last page of your assignment)
\addinformationsheet
{\Large
\textbf{Your name:} Luis Perez  % Put your name here
\\
\textbf{Email:} luis0@stanford.edu \hspace*{7cm}  % Put your e-mail here
\textbf{SUID:} 05794739  % Put your student ID here
\\*[2ex] 
}
Discussion Group: None  % List your study group here
\\
\vfill\vfill
I acknowledge and accept the Honor Code.\\*[3ex]
\bigskip
\textit{(Signed)} 
LAP
\vfill





\end{document}
