\documentclass[11pt]{article}
\usepackage{amsmath}
\usepackage{amssymb}
\usepackage{graphicx}
\usepackage{fancyhdr}
\usepackage{enumerate}
\usepackage{titlesec}
\usepackage[colorlinks=true,urlcolor=blue]{hyperref}

\titlespacing{\subsubsection}{0pt}{0pt}{0pt}

% No page numbers
%\pagenumbering{gobble}

% INFORMATION SHEET (DO NOT EDIT THIS PART) ---------------------------------------------
\newcommand{\addinformationsheet}{
\clearpage
\thispagestyle{empty}
\begin{center}
\LARGE{\bf \textsf{Information sheet\\CS224W: Analysis of Networks}} \\*[4ex]
\end{center}
\vfill
\textbf{Assignment Submission } Fill in and include this information sheet with each of your assignments.  This page should be the last page of your submission.  Assignments are due at 11:59pm and are always due on a Thursday.  All students (SCPD and non-SCPD) must submit their homeworks via GradeScope (\url{http://www.gradescope.com}). Students can typeset or scan their homeworks. Make sure that you answer each (sub-)question on a separate page. That is, one answer per page regardless of the answer length. Students also need to upload their code at \url{http://snap.stanford.edu/submit}. Put all the code for a single question into a single file and upload it. Please do not put any code in your GradeScope submissions. 
\\
\\
\textbf{Late Homework Policy } Each student will have a total of {\em two} free late periods. {\em Homeworks are due on Thursdays at 11:59pm PDT and one late period expires on the following Monday at 11:59pm PDT}.  Only one late period may be used for an assignment.  Any homework received after 11:59pm PDT on the Monday following the homework due date will receive no credit.  Once these late periods are exhausted, any assignments turned in late will receive no credit.
\\
\\
\textbf{Honor Code } We strongly encourage students to form study groups. Students may discuss and work on homework problems in groups. However, each student must write down their solutions independently i.e., each student must understand the solution well enough in order to reconstruct it by him/herself.  Students should clearly mention the names of all the other students who were part of their discussion group. Using code or solutions obtained from the web (github/google/previous year solutions etc.) is considered an honor code violation. We check all the submissions for plagiarism. We take the honor code very seriously and expect students to do the same. 
\vfill
\vfill
}
% ------------------------------------------------------------------------------

% MARGINS (DO NOT EDIT) ---------------------------------------------
\oddsidemargin  0.25in \evensidemargin 0.25in \topmargin -0.5in
\headheight 0in \headsep 0.1in
\textwidth  6.5in \textheight 9in
\parskip 1.25ex  \parindent 0ex \footskip 20pt
% ---------------------------------------------------------------------------------

% HEADER (DO NOT EDIT) -----------------------------------------------
\newcommand{\problemnumber}{0}
\newcommand{\myname}{name}
\newfont{\myfont}{cmssbx10 scaled 1000}
\pagestyle{fancy}
\fancyhead{}
\fancyhead[L]{\myfont Question \problemnumber, Problem Set 1, CS224W}
%\fancyhead[R]{\bssnine \myname}
\newcommand{\newquestion}[1]{
\clearpage % page break and flush floats
\renewcommand{\problemnumber}{#1} % set problem number for header
\phantom{}  % Put something on the page so it shows
}
% ---------------------------------------------------------------------------------


% BEGIN HOMEWORK HERE
\begin{document}

\graphicspath{ {../../code/output/} }
% Question 1.1
\newquestion{1.1}
The collaboration network has a higher proportion of both low-degree and high-degree nodes as compared to both of the random network models (see Figure \ref{fig:random_network_distribution}). Neither the Erdos-Renyi models nor the Small World model have any nodes with degrees larger than ~15, and while the Erdo-Renyi models has some low-degree nodes (<5), the proportion is still an order of magnitude different from the collaboration network.

\begin{figure}[h!]
\centering
\includegraphics[width=0.5\textwidth]{erdo_small_collab_log_logdegree_distribution.png}
\caption{Log-Log Plot of Degree Distribution}
\label{fig:random_network_distribution}
\end{figure}

% Question 1.2
\newquestion{1.2}

\subsubsection*{(a)}
We show how to compute the excess degree distribution $\{q_k\}$ given only the degree distribution $\{p_k\}$. The key is in understanding how the degree distribution is calculated. We have:
$$
p_k = \frac{p_k'}{\sum_i p_i'} = \frac{p_k'}{N}
$$
where
$$
p_k' = \sum_{i \in V} I_{N(i) = k}
$$
and $N(i)$ is the number of neighbors of node $i$ and $N$ the total number of nodes. We can therefore express $q_k'$ as:
\begin{align}
q_k' &= \sum_{i \in V} \sum_{(i,j) \in E} I_{N(j) = k + 1} \\
&= \sum_{i \in V} \sum_{(i,j) \in E} I_{N(i) = k+1} \tag{graph in undirected, $(u,v) \equiv (v,u)$} \\
&= \sum_{i \in V} I_{N(i) = k + 1} \sum_{(i, j) \in E} 1 \tag{I does not depend on inner sum over $j$}\\
&= \sum_{i \in V} I_{N(i) = k + 1} N(i) \tag{Sum over $j$ just counts the neighbors of $i$} \\
&= (k+1) \sum_{i\in V} I_{N(i) = k + 1} \tag{Non-zero only if $N(i) = k+1$} \\
&= (k+1)p_{k+1}'
\end{align}
The key in the above argument is to notice that the definition of $q_k'$ is to look at every edge in the graph once and count the number of "endpoints" with degree $k+1$. Since in an undirected graph the existence of $(u,v)$ with endpoint $v$ implies the existence of $(v,u)$ with endpoint $u$, the computation is equivalent to counting the "startpoints" with degree $k+1$ (since every endpoint is also a startpoint for the reverse edge).

The above implies the following:
\begin{align}
q_k &= \frac{q_k'}{\sum_i q_i'} \\
&= \frac{(k+1)p_{k+1}'}{\sum_i q_i'} \\
&= \frac{(k+1)p_{k+1}\sum_i p_i'}{\sum_i q_i;} \\
&= \frac{(k+1)Np_{k+1}}{\sum_i q_i'}
\end{align}
\subsubsection*{(b)}
We plot the excess degree distribution for all our models:

\begin{figure}[h!]
\centering
\includegraphics[width=0.5\textwidth]{erdo_small_collab_log_log_excess_degree_distribution.png}
\caption{Log-Log Plot of Degree Distribution}
\label{fig:random_network_distribution}
\end{figure}

The key difference between the excess degree and the degree distribution of the collaboration network is that the excess degree distribution is more flat, in the sense that the probability of a low excess degree is lower than the probability of a low degree and the probability of a high excess degree is higher than the probability of a high degree. Intuitively, this means that if we choose a random collaboration, the endpoints have a higher likelihood to have a large degree than we would expect from just the degree distribution.

The interesting thing is that the generated networks do not show this effect to the same magnitude, as we can see from the expected degree and expected excess degree:

\begin{table}[h!]
	\centering
  \begin{tabular}{||c | c | c ||} 
		\hline
		Network & Expected Degree & Expected Excess Degree  \\ 
		\hline\hline
		Erdos-Renyi & 5.526135 & 5.542323 \\ 
		\hline
		Small World & 5.526135 & 4.800400 \\
		\hline
		Collaboration & 5.526135 &  15.870409 \\
		\hline
	\end{tabular}
	\caption{Expected Degree and Expceted Excess Degree}
\end{table}


% Question 1.3
\newquestion{1.3}
We calculate the clustering coefficients, which are:

\begin{table}[h!]
	\centering
  \begin{tabular}{||c | c ||} 
		\hline
		Network & Clustering Coefficient  \\ 
		\hline\hline
		Erdos-Renyi & 0.001102 \\ 
		\hline
		Small World & 0.284075 \\
		\hline
		Collaboration & 0.529636 \\
		\hline
	\end{tabular}
	\caption{Clustering Coefficient for Different Models}
\end{table}

The largest clustering coefficient belongs to the Collaboration Network.

If we consider the fact that many papers often time have a large number of authors, and that, in isolation, the clustering coefficient of all of these nodes would be 1 (since every edge exists between all authors that are listed on the same paper), the 0.529636 coefficient does not seem unreasonable. It is likely that authors continue to collaborate with a similar set of people and, at some point, some paper is published which includes the majority of the collaborators as authors, thereby increasing the clustering coefficient of that group.


% Question 2.1
\newquestion{2.1}

% Question 2.2
\newquestion{2.2}

% Question 2.3
\newquestion{2.3}

% Question 2.3
\newquestion{2.4}

% Question 3.1
\newquestion{3.1}

\subsubsection*{(a)}

\subsubsection*{(b)}

\subsubsection*{(c)}

\subsubsection*{(d)}

\subsubsection*{(e)}

% Question 3.2
\newquestion{3.2}


% Information sheet
% Fill out the information below (this should be the last page of your assignment)
\addinformationsheet
{\Large
\textbf{Your name:} \hrulefill  % Put your name here
\\
\textbf{Email:} \underline{\hspace*{7cm}}  % Put your e-mail here
\textbf{SUID:} \hrulefill  % Put your student ID here
\\*[2ex] 
}
Discussion Group: \hrulefill   % List your study group here
\\
\vfill\vfill
I acknowledge and accept the Honor Code.\\*[3ex]
\bigskip
\textit{(Signed)} 
\hrulefill   % Replace this line with your initials
\vfill





\end{document}
