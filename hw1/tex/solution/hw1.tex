\documentclass[11pt]{article}
\usepackage{amsmath}
\usepackage{amssymb}
\usepackage{graphicx}
\usepackage{fancyhdr}
\usepackage{enumerate}
\usepackage{titlesec}
\usepackage[colorlinks=true,urlcolor=blue]{hyperref}

\titlespacing{\subsubsection}{0pt}{0pt}{0pt}

% No page numbers
%\pagenumbering{gobble}

% INFORMATION SHEET (DO NOT EDIT THIS PART) ---------------------------------------------
\newcommand{\addinformationsheet}{
\clearpage
\thispagestyle{empty}
\begin{center}
\LARGE{\bf \textsf{Information sheet\\CS224W: Analysis of Networks}} \\*[4ex]
\end{center}
\vfill
\textbf{Assignment Submission } Fill in and include this information sheet with each of your assignments.  This page should be the last page of your submission.  Assignments are due at 11:59pm and are always due on a Thursday.  All students (SCPD and non-SCPD) must submit their homeworks via GradeScope (\url{http://www.gradescope.com}). Students can typeset or scan their homeworks. Make sure that you answer each (sub-)question on a separate page. That is, one answer per page regardless of the answer length. Students also need to upload their code at \url{http://snap.stanford.edu/submit}. Put all the code for a single question into a single file and upload it. Please do not put any code in your GradeScope submissions. 
\\
\\
\textbf{Late Homework Policy } Each student will have a total of {\em two} free late periods. {\em Homeworks are due on Thursdays at 11:59pm PDT and one late period expires on the following Monday at 11:59pm PDT}.  Only one late period may be used for an assignment.  Any homework received after 11:59pm PDT on the Monday following the homework due date will receive no credit.  Once these late periods are exhausted, any assignments turned in late will receive no credit.
\\
\\
\textbf{Honor Code } We strongly encourage students to form study groups. Students may discuss and work on homework problems in groups. However, each student must write down their solutions independently i.e., each student must understand the solution well enough in order to reconstruct it by him/herself.  Students should clearly mention the names of all the other students who were part of their discussion group. Using code or solutions obtained from the web (github/google/previous year solutions etc.) is considered an honor code violation. We check all the submissions for plagiarism. We take the honor code very seriously and expect students to do the same. 
\vfill
\vfill
}
% ------------------------------------------------------------------------------

% MARGINS (DO NOT EDIT) ---------------------------------------------
\oddsidemargin  0.25in \evensidemargin 0.25in \topmargin -0.5in
\headheight 0in \headsep 0.1in
\textwidth  6.5in \textheight 9in
\parskip 1.25ex  \parindent 0ex \footskip 20pt
% ---------------------------------------------------------------------------------

% HEADER (DO NOT EDIT) -----------------------------------------------
\newcommand{\problemnumber}{0}
\newcommand{\myname}{name}
\newfont{\myfont}{cmssbx10 scaled 1000}
\pagestyle{fancy}
\fancyhead{}
\fancyhead[L]{\myfont Question \problemnumber, Problem Set 1, CS224W}
%\fancyhead[R]{\bssnine \myname}
\newcommand{\newquestion}[1]{
\clearpage % page break and flush floats
\renewcommand{\problemnumber}{#1} % set problem number for header
\phantom{}  % Put something on the page so it shows
}
% ---------------------------------------------------------------------------------


% BEGIN HOMEWORK HERE
\begin{document}

\graphicspath{ {../../code/output/} }
% Question 1.1
\newquestion{1.1}
The collaboration network has a higher proportion of both low-degree and high-degree nodes as compared to both of the random network models (see Figure \ref{fig:random_network_distribution}). Neither the Erdos-Renyi models nor the Small World model have any nodes with degrees larger than ~15, and while the Erdo-Renyi models has some low-degree nodes (<5), the proportion is still an order of magnitude different from the collaboration network.

\begin{figure}[h!]
\centering
\includegraphics[width=0.5\textwidth]{erdo_small_collab_log_logdegree_distribution.png}
\caption{Log-Log Plot of Degree Distribution}
\label{fig:random_network_distribution}
\end{figure}

% Question 1.2
\newquestion{1.2}

\subsubsection*{(a)}
We show how to compute the excess degree distribution $\{q_k\}$ given only the degree distribution $\{p_k\}$. The key is in understanding how the degree distribution is calculated. We have:
$$
p_k = \frac{p_k'}{\sum_i p_i'} = \frac{p_k'}{N}
$$
where
$$
p_k' = \sum_{i \in V} I_{N(i) = k}
$$
and $N(i)$ is the number of neighbors of node $i$ and $N$ the total number of nodes. We can therefore express $q_k'$ as:
\begin{align}
q_k' &= \sum_{i \in V} \sum_{(i,j) \in E} I_{N(j) = k + 1} \\
&= \sum_{i \in V} \sum_{(i,j) \in E} I_{N(i) = k+1} \tag{graph in undirected, $(u,v) \equiv (v,u)$} \\
&= \sum_{i \in V} I_{N(i) = k + 1} \sum_{(i, j) \in E} 1 \tag{I does not depend on inner sum over $j$}\\
&= \sum_{i \in V} I_{N(i) = k + 1} N(i) \tag{Sum over $j$ just counts the neighbors of $i$} \\
&= (k+1) \sum_{i\in V} I_{N(i) = k + 1} \tag{Non-zero only if $N(i) = k+1$} \\
&= (k+1)p_{k+1}'
\end{align}
The key in the above argument is to notice that the definition of $q_k'$ is to look at every edge in the graph once and count the number of "endpoints" with degree $k+1$. Since in an undirected graph the existence of $(u,v)$ with endpoint $v$ implies the existence of $(v,u)$ with endpoint $u$, the computation is equivalent to counting the "startpoints" with degree $k+1$ (since every endpoint is also a startpoint for the reverse edge).

The above implies the following:
\begin{align}
q_k &= \frac{q_k'}{\sum_i q_i'} \\
&= \frac{(k+1)p_{k+1}'}{\sum_i q_i'} \\
&= \frac{(k+1)p_{k+1}\sum_i p_i'}{\sum_i q_i;} \\
&= \frac{(k+1)Np_{k+1}}{\sum_i q_i'}
\end{align}
\subsubsection*{(b)}
We plot the excess degree distribution for all our models:

\begin{figure}[h!]
\centering
\includegraphics[width=0.5\textwidth]{erdo_small_collab_log_log_excess_degree_distribution.png}
\caption{Log-Log Plot of Degree Distribution}
\end{figure}

The key difference between the excess degree and the degree distribution of the collaboration network is that the excess degree distribution is more flat, in the sense that the probability of a low excess degree is lower than the probability of a low degree and the probability of a high excess degree is higher than the probability of a high degree. Intuitively, this means that if we choose a random collaboration, the endpoints have a higher likelihood to have a large degree than we would expect from just the degree distribution.

The interesting thing is that the generated networks do not show this effect to the same magnitude, as we can see from the expected degree and expected excess degree:

\begin{table}[h!]
	\centering
  \begin{tabular}{||c | c | c ||} 
		\hline
		Network & Expected Degree & Expected Excess Degree  \\ 
		\hline\hline
		Erdos-Renyi & 5.526135 & 5.542323 \\ 
		\hline
		Small World & 5.526135 & 4.800400 \\
		\hline
		Collaboration & 5.526135 &  15.870409 \\
		\hline
	\end{tabular}
	\caption{Expected Degree and Expceted Excess Degree}
\end{table}


% Question 1.3
\newquestion{1.3}
We calculate the clustering coefficients, which are:

\begin{table}[h!]
	\centering
  \begin{tabular}{||c | c ||} 
		\hline
		Network & Clustering Coefficient  \\ 
		\hline\hline
		Erdos-Renyi & 0.001102 \\ 
		\hline
		Small World & 0.284075 \\
		\hline
		Collaboration & 0.529636 \\
		\hline
	\end{tabular}
	\caption{Clustering Coefficient for Different Models}
\end{table}

The largest clustering coefficient belongs to the Collaboration Network.

If we consider the fact that many papers often time have a large number of authors, and that, in isolation, the clustering coefficient of all of these nodes would be 1 (since every edge exists between all authors that are listed on the same paper), the 0.529636 coefficient does not seem unreasonable. It is likely that authors continue to collaborate with a similar set of people and, at some point, some paper is published which includes the majority of the collaborators as authors, thereby increasing the clustering coefficient of that group.


% Question 2.1
\newquestion{2.1}
For this question, let us consider the behavior for a node that belongs to each of SCC, IN, or OUT.
\begin{itemize}
\item SCC: A node in the SCC would reach all nodes in SCC + OUT in a forward pass, and all nodes in SCC + IN in a backwards pass. Since we expect a good portion of the nodes to belong to the SCC, each of these values should be relatively ``good'' ($> 15\%$) chunck of the graph.
\item IN: A node in the IN would reach all nodes in SCC + OUT + $\epsilon_1$ in a forward pass, and $\epsilon_2$ nodes in the backward pass for some small $\epsilon_i$.
\item OUT: A node int he OUT would reach $\epsilon_1$ nodes in the forward pass, and SCC + IN + $\epsilon_2$ in the backwards pass for some small $\epsilon_i$.
\end{itemize}

Our conlusions are therefore (see code for more exact numbers) as well as percentage of nodes in largest SCC:
\begin{itemize}
\item Node 1952 in the epinions graph belongs to the IN region. A forward pass (SCC + OUT + $\epsilon$) hits $~62.83\%$ of nodes and backward pass only hits $~0.001\%$ of nodes.
\item Node 9809 in the epinions graph belongs to the OUT region. A forward pass only hits $~0.001\%$ of nodes and a backward pass (SCC + IN + $\epsilon$) hits $~74.41\%$ of nodes. This is further corraborated when we sample a node in the SCC for which the forward pass (SCC + OUT) hits $~62.83\%$ and the backward pass (SCC + IN) hits $~74.41\%$ of nodes.
\item Node 189587 in the email graph belongs to the SCC. A forward pass (SCC + OUT) hits $~19.65\%$ of nodes and backward pass (SCC + IN) hits $~69.84\%$ of nodes. 
\item Node 675 in the email graph belongs to the OUT region. A forward pass only hits only $~0.0003\%$ of nodes and a backward pass (SCC + IN + $\epsilon$) hits $~69.85\%$, as expected. Furthermore, we located a node $0$ which we classify as belonging to IN, since a forward pass (SCC + OUT + $\epsilon$) hits $~19.65\%$ of nodes and a backward pass only $~0.0003\%$. This matches are previous proposals. 
\end{itemize}

% Question 2.2
\newquestion{2.2}
For details on further analysis, see the provided code.

We present the cdf for our random start bfs for the Email network:
\begin{figure}[h!]
\centering
\includegraphics[width=0.5\textwidth]{bfs_cdf_forward_email_graph.png}
\caption{CDF of Forward BFS on randomly selected nodes from Email Network}
\end{figure}
\begin{figure}[h!]
\centering
\includegraphics[width=0.5\textwidth]{bfs_cdf_backward_email_graph.png}
\caption{CDF of Backward BFS on randomly selected nodes from Email Network}
\end{figure}

The BFS traversal behavior appears to imply that we have a bowtie structure, because a very large percentage of the randomly chosen nodes ``exploded'' when following forward links (51\%), 19\% ``exploded'' when following both forward and back links, and a small minority (7\%) when following backward links. The above also implies that we have a very large IN region (half of nodes), a relatively large SCC (1/5 of nodes), a small OUT region, with a somewhat large number of nodes completely disconnected from the graph (~23\%) or forming very small tendrils.

\newpage
We present the cfg for our random start bfs for the Epinions network:
\begin{figure}[h!]
\centering
\includegraphics[width=0.5\textwidth]{bfs_cdf_forward_epinions_graph.png}
\caption{CDF of Forward BFS on randomly selected nodes from Epinions Network}
\end{figure}
\begin{figure}[h!]
\centering
\includegraphics[width=0.5\textwidth]{bfs_cdf_backward_epinions_graph.png}
\caption{CDF of Backward BFS on randomly selected nodes from Epinions Network}
\end{figure}

The BFS traversal here also appears to imply a bowtie structure, though somewhat more complicated IN/OUT regions since it appears that it can take up to 10 steps to go from IN $\to$ SCC and up to 3 to go from OUT $\to$ SCC, so these regions appear more elongated. We still have, however, a large (39\%) SCC, a large (20\%) OUT region, a large (36\%) IN region, and a small percentage of short tendrils or possibly disconnected regions (5\%).

% Question 2.3
\newquestion{2.3}
We present the sizes of each region.

\begin{table}[h!]
	\centering
  \begin{tabular}{||c | c | c | c | c | c | c ||} 
		\hline
		Network & Total & SCC & IN & OUT & TENDRILS & DISCONNECTED  \\ 
		\hline\hline
		Email & 265214 & 34203 & 151023 & 17900 & 21706 & 40382 \\ 
		\hline
		Epinions & 75879 & 32223 & 24236 & 15453 & 3965 & 2 \\
		\hline
	\end{tabular}
	\caption{Size of Regions in Network}
\end{table}

The sizes were computed in a straigh-forward manner. We can compute the largest WCC using `snap.GetMxWcc' and obtain $N_{Wcc}$ and we can also compute the larges SCC using `snap.GetMxScc' and obtain $N_{Scc}$. Let $N_{forward}$ be the number of nodes found in a forward bfs starting at an SCC and similarly calculate $N_{backward}$. Then we have:
\begin{align*}
N_{In} &= N_{backward} - N_{Scc} \\
N_{Out} &= _{forward} - N_{Scc} \\
N_{Disconnected} &= N - N_{Wcc} \\
N_{Tendrils} &= N_{Wcc} - (N_{In} + N_{Out} + N_{Scc})
\end{align*}

% Question 2.3
\newquestion{2.4}
The result from one of 3 trials using 1000 pairs.

Probability of path for entire Epinions is 45.8\%.
Probability of path in largest WCC of Epinions is 46.1\%.
Probability of path for entire Email is 13.0\%.
Probability of path in largest WCC of Email is 19.9\%.

We see that the probability of a path from Epinions is 2.5x higher that for the Email network, which aligns with our previous findings since the Epinions network has a much larger SCC and despite having a large IN, the email network has many disconnected components and the IN is large number of randonly chosen nodes to be both in IN or the endpoint in IN. Furthermore, we see that when we focus just on the WCC, the effect is larger in the Email network, very likely due to the fact that the disconnected region of the Epinions network is relatively small while it is relatively large in the Email network.

% Question 3.1
\newquestion{3.1}

\subsubsection*{(a)}
We show that given a value $d$ and a network node $v$, there are $b^d - b^{d-1}$ network nodes satisfying $h(u,w) = d$.

We wish to count all $w$ that satisfy $h(u,w) = d$. By definition, this is all leaves of $T$ such that the height of the subtree $T'$ rooted at the least common ancestor of $u$ and $w$ is $d$. Given that $T'$ is also a complete, perfectly balanced $b$-ary tree (by property of subtrees), we have the $h(T') = d = \log_b N_{T'} \implies N_{T'} = b^d$. Therefore $T'$ has exactly $b^d$ network nodes. However, one branch of $T'$ must contain $u$, and the LCA of the network nodes in this branch is not the root of $T'$, since at the very least, the child which forms the root of the subtree ($T''$) containing $u$ is a closer ancestor. Note that $T''$ is again a complete, perfectly balanced $b$-ary tree of height $d-1$, therefore it contains $b^{d-1}$ network nodes whose distance from $u < d$.

Finally, consider any network nodes outside this subtree $T'$, which implies the shared ancestor between them and $u$ is a parent of $T'$. Therefore, the distance from $u > d$.

By the above, we have that there are exactly $b^d - b^{d-1}$ network nodes satisfying $h(u,w) = d$.

\subsubsection*{(b)}
W show that $Z \leq \log_b N$.

We recall that for a node $v$, $Z = \sum_{w \neq v} b^{-h(v,w)}$. Note that by summetry, $Z$ is the same for all nodes $v$. Then rather than sum over the network nodes $w$, we sum over the distances $h(v,w)$ and use the results from part (a). We have:
\begin{align*}
Z &= \sum_{w \neq v} b^{-h(v,w)} \\
&= \sum_{d=1}^{\log_b N} (\text{\# nodes with distance d from v}) * b^{-d} \\
&= \sum_{d=1}^{\log_b N} (b^d - b^{d-1})b^{-d} \tag{results from (a)} \\
&= \sum_{d=1}^{\log_b N} (1 - b^{-1}) \\
&= (\log_b N)(1 - b^{-1}) \\
&\leq \log_b N
\end{align*}

\subsubsection*{(c)}

For two leaf nodes $v$ and $t$, let $T'$ be the subtree of $$L(v, t)$$ satisfying:
\begin{itemize}
\item $T'$ is of height $h(v, t) − 1$
\item $T'$ contains $t$
\item $T'$ does not contain $v$.
\end{itemize}

Consider an edge $e$ from $v$ to a random node $u$ sampled from $p_v$. We say that $e$ points to $T'$ if $u$ is a leaf node of $T'$ We show that the probability of e pointing to $T'$ is at least $\frac{1}{b \log_b N}$.

Let $p_v(T')$ be the probability we're interested in. Then note the following:
\begin{align*}
p_v(T') &= \sum_{w \in T'} p_v(w) \tag{events are independent} \\
&= \frac{1}{Z}\sum_{w \in T'} b^{-h(v,w)} \tag{definition of $p_v(w)$} \\
&\geq \frac{1}{\log_b N} \sum_{w \in T'} b^{-h(v,w)} \tag{results from (b)} \\
&\geq \frac{1}{\log_b N} \sum_{w \in T'} b^{-h(v,t)} \tag{all $w$ belong to $T'$ so they share a common ancestor at $h(v,t)$ height, at least} \\
&= \frac{1}{\log_b N} b^{h(v,t) - 1} b^{-h(v,t)} \tag{$T'$ has height $h(v,t) - 1$} \\
&= \frac{1}{b \log_b N}
\end{align*}

\subsubsection*{(d)}
Let the out-degree $k$ for each node be $c(\log_b N)^2$ for some constant $c$. We show that the probability that $v$ has no edge pointing to $T'$ ($p_v(\bar{T'})$) is at most $N^{−\Theta}$ where $\Theta = \frac{c}{b}$. Consider the below:

\begin{align*}
p_v(\bar{T'}) &= \left(1 - p_v(T')\right)^k \tag{all $k$ out edges must fail to reach $T'$} \\
&\leq e^{-kp_v(T')} \tag{$1 + x \leq e^x, \forall x \in \mathbb{R}$}\\
&\leq e^{-\frac{c}{b}\log_b N} \tag{results from (c) and using $k = c(\log_b N)^2$} \\
&= N^{-\frac{c}{b}} \tag{properties of exponents}
\end{align*}

Note that the result above implies that given two nodes $v,t$ in a tree, the probability that $v$ has an edge to $T'$ is at least $1 - N^{-\frac{c}{b}}$, which for large enough $N$ is $1$. In other words, for any node $v$, with high-probability, there exists an edge $e$ from $v$ to $u$ such that $h(u,t) < h(v, t)$ 

\subsubsection*{(e)}
We show that starting from any (leaf) node $s$, we can reach any (leaf) node $t$ within $O(\log_b N)$ steps.

As we have argued in (d), with high probability ($1 - N^{-\frac{c}{b}}$), for any (leaf) node $v$ we can get to a (leaf) node $u$ satisfying $h(u,t) < h(v,t)$. Note the strict inequality. Then let us start with $v=s=s_0$, and by applying the argument, we can find some $s_1$ such that $h(s_1, t) < h(s_0,t)$. Continuing as such iteratively, we can find $\{s_i\}$ such that $0 < h(s_T, t) \cdots < h(s_i, t) < \cdots < h(s_1, t) < h(s_0, t)$ where each step is taken with high probability $(1 - N^{-\frac{c}{b}})$. Furthermore, note that $T = O(\log_b N)$ since $h(v, t) \leq \log_b N$ ($=$ in the case that $v$ and $t$ only share the root node of the tree), and each step of the process must decrease the distance by at least $1$ (due to strict inequality). 

Lastly, we note that $\lim_{N \to \infty} \left(1 - N^{-\frac{c}{b}} \right)^{\log_{b} N} = 1$.

% Question 3.2
\newquestion{3.2}
We run experiments of the above and show plots.

\begin{figure}[h!]
\centering
\includegraphics[width=0.5\textwidth]{average_path_length_vs_alpha.png}
\caption{Experiment Results: Average Path Lengh of Successful Search vs Alpha Value}
\label{fig:path_length}
\end{figure}

\begin{figure}[h!]
\centering
\includegraphics[width=0.5\textwidth]{average_success_vs_alpha.png}
\caption{Experiment Results: Success Rate of Search vs Alpha Value}
\label{fig:success_rate}
\end{figure}


% Information sheet
% Fill out the information below (this should be the last page of your assignment)
\addinformationsheet
{\Large
\textbf{Your name:} \hrulefill  % Put your name here
\\
\textbf{Email:} \underline{\hspace*{7cm}}  % Put your e-mail here
\textbf{SUID:} \hrulefill  % Put your student ID here
\\*[2ex] 
}
Discussion Group: \hrulefill   % List your study group here
\\
\vfill\vfill
I acknowledge and accept the Honor Code.\\*[3ex]
\bigskip
\textit{(Signed)} 
\hrulefill   % Replace this line with your initials
\vfill





\end{document}
